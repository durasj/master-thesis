\includepdf{./assets/CertificationOfAuthorship.pdf}

% Parts before the first chapter should have roman literals
\pagenumbering{roman}
\setcounter{page}{1}

\rhead{}

\chapter*{Acknowledgements}
\addcontentsline{toc}{chapter}{Acknowledgements}

First of all, I would like to thank my advisor, Dr. Frank Glavin, and the university staff, who patiently answered my questions and provided me with helpful feedback.
Then, I would like to thank my brother Jan Duras for his help with the graphic design and adaptation of the learning material.
Moreover, I would like to thank everybody who generously offered their time to participate in the comparative study.
Last but not least, I would like to thank my family, friends, and collegues, who encouraged me and gave me the time and space needed to focus on this thesis.

\glsunsetall % Glossary terms should be used in short form in lists

\tableofcontents

\listoftables

\listoffigures

\glsresetall % Glossary terms used in lists should have no influence on the rest

\printglossary[type=\acronymtype, style=long, nonumberlist, title=List of Acronyms]

\chapter*{Abstract}
\addcontentsline{toc}{chapter}{Abstract}

Older desktop learning tools pose a challenge in a world that increasingly relies on new classes of devices such as smartphones, Chromebooks, and tablets.
While the use of new classes of devices is mostly a matter of preference in developed countries, households in developing countries often do not own any other device capable of running desktop applications.
This thesis explores the feasibility of efficiently reimplementing the functionality of a popular desktop learning tool as a modern web application accessible to a wide range of users.
Additionally, it uses remote moderated usability testing to perform a between-subjects comparison of the existing and proposed software, focusing on quantitative data.
The proposed software (\href{https://chipsandcode.com}{chipsandcode.com}) mitigates common disadvantages of the web while achieving efficiency improvements (presented at \href{https://thesis.chipsandcode.com}{thesis.chipsandcode.com}).
Data from the conducted comparative study show participants ($n=12$ per group) completed the work in one-third of the time ($512 \pm190 s$ vs $1497 \pm326 s$, $\alpha=0.05$, $p<0.001$) and experienced considerably fewer confusing situations ($p<0.001$).
Considering common reasons for Massive Open Online Course (MOOC) dropouts are lack of time and technical problems, more efficient and less confusing learning tools could be of particular interest.
Moreover, web-based learning tools could be easier to incorporate into various learning materials using embedding or linking, particularly if they are easier to pick up.

\cleardoublepage
\rhead{\nouppercase{\rightmark}}

% The rest has normal arabic numbering
\pagenumbering{arabic}
\setcounter{page}{1}
