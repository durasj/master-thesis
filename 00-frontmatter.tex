\includepdf{./assets/CertificationOfAuthorship.pdf}

% Parts before the first chapter should have roman literals
\pagenumbering{roman}
\setcounter{page}{1}

\rhead{}

\chapter*{Acknowledgements}
\addcontentsline{toc}{chapter}{Acknowledgements}

First of all, I would like to thank my advisor Dr Frank Glavin and university staff who patiently answered my questions and provided me with helpful feedback.
Secondly, I would like to thank my brother Jan Duras for the help with the graphic design and adaptation of Nand2Tetris learning material.
Moreover, I would like to thank everybody who generously offered their time to participate in the comparative study.
Last but not least, I would like to thank my family, friends, and collegues, who encouraged me and gave me the time and space needed to focus on this thesis.

\glsunsetall % Glossary terms should be used in short form in lists

\tableofcontents

\listoftables

\listoffigures

\glsresetall % Glossary terms used in lists should have no influence on the rest

\printglossary[type=\acronymtype, style=long, nonumberlist, title=List of Acronyms]

\chapter*{Abstract}
\addcontentsline{toc}{chapter}{Abstract}

Basic low-level computing concepts can be beneficial to people from all walks of life - from the general public curious about computers through professional software engineers without formal education to Computer Science students.
One of the most popular relevant learning materials is the Nand2Tetris course used by hundreds of institutions globally.
However, it is supported by an older desktop learning tool.
Considering there is a trend of increased reliance on smartphones in the age group 18-29 and the K12 market is saturated with Chromebooks and tablets, we may be seeing a new generation that will be underserved.
While the use of new classes of devices is mostly a matter of preference in developed countries, households in developing countries often do not own any device that would be capable of running desktop applications.
This thesis shows that it is feasible to efficiently reimplement the functionality of the desktop tool as a modern web application that is accessible from all types of devices.
The proposed implementation mitigates common disadvantages of the web while achieving improvements in efficiency.
The conducted comparative study shows X and Y ().
Considering common reasons for MOOC dropouts are lack of time and technical problems, more efficient and less confusing learning tools could be beneficial.
Moreover, web-based learning tools can be embedded within the learning content and reused in more materials if they are easier and faster to pick up.
Further research is needed (usability), and to confirm the observed link between time on task and sus score.
Also, more work is needed to see if these results can be reproduced with other learning tool from other fields of study.

\cleardoublepage
\rhead{\nouppercase{\rightmark}}

% The rest has normal arabic numbering
\pagenumbering{arabic}
\setcounter{page}{1}
