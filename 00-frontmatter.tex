\includepdf{./assets/CertificationOfAuthorship.pdf}

% Parts before the first chapter should have roman literals
\pagenumbering{roman}
\setcounter{page}{1}

\rhead{}

\chapter*{Acknowledgements}
\addcontentsline{toc}{chapter}{Acknowledgements}

First of all, I would like to thank my advisor, Dr. Frank Glavin, and the university staff, who patiently answered my questions and provided me with helpful feedback.
Then, I would like to thank my brother Jan Duras for his help with the graphic design and adaptation of the learning material.
Moreover, I would like to thank everybody who generously offered their time to participate in the comparative study.
Last but not least, I would like to thank my family, friends, and collegues, who encouraged me and gave me the time and space needed to focus on this thesis.

\glsunsetall % Glossary terms should be used in short form in lists

\tableofcontents

\listoftables

\listoffigures

\glsresetall % Glossary terms used in lists should have no influence on the rest

\printglossary[type=\acronymtype, style=long, nonumberlist, title=List of Acronyms]

\chapter*{\vspace{-40pt}Abstract}
\addcontentsline{toc}{chapter}{Abstract}

One of the most popular learning materials on low-level computer principles is the Nand2Tetris course, used globally by hundreds of organisations.
However, like many materials, it is supported by an older desktop learning tool.
Considering the increased reliance on smartphones in the age group 18--29 and the K12 market saturated with Chromebooks and tablets, we may see an underserved generation.
While the use of new classes of devices is mostly a matter of preference in developed countries, households in developing countries often do not own any other device that would be capable of running desktop applications.
This thesis shows that it is feasible to efficiently reimplement the functionality of the desktop tool as a modern web application accessible from all types of devices.
The proposed \href{https://chipsandcode.com}{chipsandcode.com} mitigates common disadvantages of the web while achieving efficiency improvements presented at \href{https://thesis.chipsandcode.com}{thesis.chipsandcode.com}.
Data from the conducted comparative study show participants ($n=12$ per group) completed the work in $1/3$ of the time ($512 \pm190 s$ vs $1497 \pm326 s$, $\alpha=0.05$, $p<0.001$) and experienced considerably fewer confusing situations ($p<0.001$).
Considering common reasons for MOOC dropouts are lack of time and technical problems, more efficient and less confusing learning tools could be specifically interesting.
Moreover, web-based learning tools could be easier to incorporate into various learning materials using embedding or linking, particularly if they are easier to pick up.
Further research is needed to confirm the gain of 7 SUS points ($\alpha=0.1$, $p=0.22$) and to confirm the observed link between lower performance and SUS score ($\overline{r}=-.26$, $\alpha=0.05$, $p > 0.35$).
Moreover, additional work is needed to see if these results can be reproduced with other learning tools and tools from different fields of study.

\cleardoublepage
\rhead{\nouppercase{\rightmark}}

% The rest has normal arabic numbering
\pagenumbering{arabic}
\setcounter{page}{1}
