\chapter{Design and Implementation}

\todo[inline]{Contains only brief notes in the form of free-flowing text.}

Outlines considerations when migrating desktop education tool to accessible and usable, extendable, secure, and reproducible web application.
Considers the mentioned hurdles with MOOC development.

\section{Accessibility and Usability}

Designed to be usable with built-in accessibility tools shown to be used by primary and secondary beneficiaries.
Main points: semantics, ARIA.

Time to first interaction.
Minimal build size - Svelte via SvelteKit and Tailwind with three shaking and chunking.
Used bandwith and required internet connection.
Compare size and time to download the Nand2Tetris desktop (with and without JRE) app and the highly-optimised web app.
Contrary to expectations - web app does not have to be downloaded again - can be used offline as PWA.

Usable from all platforms - percentage?
Compared to desktop Java - with and without preinstalled JRE.

Progressive enhancement.
IDE - Monaco for accessibility and powerful capabilities.
Simplified contenteditable with the same syntax highlighting for mobile phones.

\section{Extensibility}

Everything (including learning materials) control versioned on Github.
Learning materials processed from Markdown files using MDsveX.

\section{Acceleration of Development}

Use of available third party libraries to accelerate development.

Nearley with Moo for simple parsers.

Use of freely available scalable services - Github, Cloudflare - CDN, CI/CD, key/value.

\section{Security}

Infrastructure and deployment: DNSSEC, HTTPS, automation - CI/CD with signed commits - open source considerations!, "Denial of Wallet".

User: Passwordless login, Google Auth

Parsers and interpreters: No eval or access to arbitrary JS APIs in interpreter.

\section{Reproducibility}

Again - everything is versioned - infrastructure, documentation, content, code.
Made for easy hosting by anybody even without the backend.
Backend optional - can be swapped for other provider.
