\chapter{Evaluation}

\todo[inline]{Outline of the covered topics will be added when finishing the chapter.}

\section{Accessibility Evaluation}
\label{Evaluation-Accessibility}

\todo[inline]{Automated and manual test results - wcag template report.}

\section{Functional Testing}
\label{Evaluation-Tests}

\todo[inline]{Unit tests - brief description of lcov results and exceptions. Example unit tests.}

\todo[inline]{List of UI tests based on use cases and showcase of some selected ones. Description of exceptions.}

\section{Comparative Usability Testing}
\label{Evaluation-UX}

The following section contains results from the comparative study.
All figures are also available at \href{https://thesis.chipsandcode.com}{thesis.chipsandcode.com} with the advantage of being interactive.

The total number of participants who joined the scheduled comparative study session was $n=26$.
Out of that, the data of two participants had to be removed - Participant $F$, who experienced technical difficulties unrelated to the software and Participant $J$, who mentioned they tried the Proposed Software in the past.
The resulting sample size that was analysed was $n=24$, with an equal distribution of $n_1=12$ and $n_2=12$ among both groups.

The list of participants and their profiles within Group A can be seen in Table~\ref{table:evaluation-profile-a}, while Group B participant profiles can be seen in Table~\ref{table:evaluation-profile-b}.

\begin{table}[H]
    \centering
    \caption{Group A participant profiles.}
    \label{table:evaluation-profile-a}
    \begin{tabular}{ l l l l l l l l }
        \rotheading{Person} & \rotheading{Prior Exp.} & \rotheading{Education} & \rotheading{Occupation} & \rotheading{Age Group} & \rotheading{Paid} & \rotheading{Intr. Motiv.} & \rotatebox{30}{\textbf{Platform}} \\ \hline
        \textbf{M} & 1 & 2 & 2 & 30-49 & 0 & 0 & Ubuntu Linux desktop \\ \hline
        \textbf{LU} & 0 & 0 & 0 & 18-29 & 1 & 1 & Windows 10 laptop \\ \hline
        \textbf{Y} & 1 & 0 & 0 & 30-49 & 0 & 2 & Windows 10 laptop \\ \hline
        \textbf{MI} & 0 & 1 & 0 & 30-49 & 0 & 0 & Ubuntu Linux laptop \\ \hline
        \textbf{MX} & 1 & 2 & 1 & 30-49 & 0 & 1 & Windows 10 desktop \\ \hline
        \textbf{R} & 0 & 2 & 0 & 18-29 & 1 & 0 & Windows 10 desktop \\ \hline
        \textbf{MK} & 0 & 1 & 2 & 30-49 & 0 & 1 & Windows 11 laptop \\ \hline
        \textbf{L} & 0 & 2 & 2 & 18-29 & 0 & 1 & macOS 13 arm64 laptop \\ \hline
        \textbf{MA} & 0 & 0 & 0 & 18-29 & 1 & 1 & Windows 10 laptop \\ \hline
        \textbf{KY} & 0 & 1 & 1 & 18-29 & 1 & 1 & Windows 10 laptop \\ \hline
        \textbf{KA} & 0 & 0 & 0 & 18-29 & 1 & 0 & Windows 7 laptop \\ \hline
        \textbf{E} & 0 & 0 & 2 & 30-49 & 1 & 2 & Windows 10 desktop \\ \hline
    \end{tabular}
\end{table}

\begin{table}[H]
    \centering
    \caption{Group B participant profiles.}
    \label{table:evaluation-profile-b}
    \begin{tabular}{ l l l l l l l l }
        \rotheading{Person} & \rotheading{Prior Exp.} & \rotheading{Education} & \rotheading{Occupation} & \rotheading{Age Group} & \rotheading{Paid} & \rotheading{Intr. Motiv.} & \rotatebox{30}{\textbf{Platform}} \\ \hline
        \textbf{P} & 0 & 1 & 1 & 18-29 & 0 & 2 & Windows 10 laptop \\ \hline
        \textbf{S} & 0 & 1 & 0 & 30-49 & 0 & 1 & Windows 10 desktop \\ \hline
        \textbf{MM} & 1 & 1 & 2 & 18-29 & 0 & 1 & Ubuntu Linux laptop \\ \hline
        \textbf{SA} & 0 & 1 & 2 & 18-29 & 0 & 0 & Windows 11 laptop \\ \hline
        \textbf{JO} & 0 & 0 & 0 & 18-29 & 1 & 0 & Windows 11 desktop \\ \hline
        \textbf{PE} & 0 & 1 & 0 & 18-29 & 1 & 1 & Windows 11 laptop \\ \hline
        \textbf{MS} & 0 & 2 & 2 & 18-29 & 0 & 1 & Windows 11 laptop \\ \hline
        \textbf{B} & 1 & 2 & 2 & 18-29 & 0 & 1 & macOS 13 arm64 laptop \\ \hline
        \textbf{V} & 0 & 2 & 2 & 18-29 & 1 & 1 & Windows 10 laptop \\ \hline
        \textbf{MH} & 1 & 2 & 2 & 30-49 & 1 & 1 & macOS 13 x86 laptop \\ \hline
        \textbf{LB} & 0 & 2 & 2 & 18-29 & 1 & 1 & Windows 10 laptop \\ \hline
        \textbf{MT} & 0 & 2 & 0 & 18-29 & 1 & 2 & Windows 10 desktop \\ \hline
    \end{tabular}
\end{table}

As we can see, despite the effort to distribute participant profiles equally, the profiles were not completely equal between the groups.
To better capture the differences, Figure~\ref{fig:plot-points} shows the cumulative "Prior Experience", "Education", "Relevant Occupation", and "Intrinsic Motivation" between the two groups.
Additionally, Figure~\ref{fig:plot-age} shows the age make-up of the groups and Figure~\ref{fig:plot-platform} platforms used by participants of each group.

\begin{figure}[H]
    \includesvg[width=380pt, keepaspectratio]{Plot_Points}
    \caption{Cumualtive profile points between groups.}
    \label{fig:plot-points}
\end{figure}

\begin{figure}[H]
    \includesvg[width=380pt, keepaspectratio]{Plot_Age}
    \caption{Age characteristics of participants.}
    \label{fig:plot-age}
\end{figure}

\begin{figure}[H]
    \includesvg[width=380pt, keepaspectratio]{Plot_Platform}
    \caption{Platforms used by participants.}
    \label{fig:plot-platform}
\end{figure}

\subsection{Efficiency}

The first kind of collected data on efficiency consisted of time-related data like the time it took to prepare the tool for use and time spent on individual tasks.
Figure~\ref{fig:plot-linear} shows a linear graph covering the mean of the time of mentioned steps between both groups.
The same data can be seen as cumulative, which is represented by Figure~\ref{fig:plot-stacked-bar}.
The total time spent on all parts was $512.42 \pm 190.01 s$ for Group A and $1,496.67 \pm 326.18 s$ for Group B ($\alpha=0.05$, $p<0.001$).
Raw data collected during the study and that served as an input to mentioned figures and results are captured in Appendix~\ref{appendix:study-time-notes}.

The second kind of data on efficiency was the number of times the participant got confused, see Figure~\ref{fig:plot-boxplot}.
For Group A, the mean was $0.58 \pm 0.42$, while for Group B it was $2.83 \pm 0.85$ ($\alpha=0.05$, $p<0.001$).

\begin{figure}[H]
    \includesvg[width=380pt, keepaspectratio]{Plot_Linear}
    \caption{Time needed to prepare software and perform tasks.}
    \label{fig:plot-linear}
\end{figure}

\begin{figure}[H]
    \includesvg[width=380pt, keepaspectratio]{Plot_StackedBar}
    \caption{Cumulative time needed to prepare software and perform tasks.}
    \label{fig:plot-stacked-bar}
\end{figure}

\begin{figure}[H]
    \includesvg[width=380pt, keepaspectratio]{Plot_BoxPlot}
    \caption{Number of times participants got confused.}
    \label{fig:plot-boxplot}
\end{figure}

\subsection{Questionnaire}

Collected responses had no coding errors, and internal reliability was acceptable at \emph{Cronbach Alpha} $\rho_T=0.708$ for Group A and $\rho_T=0.895$ for Group B.
The mean \gls{sus} score for Group A was 86.67 (97th percentile), while for Group B it was 79.58 (76th percentile).
While that represents a difference of about seven points, it did not reach desired statistical significance ($\alpha=0.1$, $p=0.22$).
Collected scores, statistical properties, and associated adjective and grade, can be seen in Figure~\ref{fig:plot-sus}.
Placement on the percentile curve is captured in Figure~\ref{fig:plot-sus-percentile}.

The time it took the participant preparing the tool and performing the assigned tasks is compared to the assigned \gls{sus} score in Figure~\ref{fig:plot-sus-time}.
While there appears to be some correlation for both Group A and Group B, at $r=-0.28$ and $r=-0.23$, respectively, the combination of the small effect size and small sample size means the likelihood of $null$-hypothesis is high, with $p > 0.35$ for both groups.

\begin{figure}[H]
    \includesvg[width=380pt, keepaspectratio]{Plot_SUS}
    \caption{\gls{sus} score comparison.}
    \label{fig:plot-sus}
\end{figure}

\begin{figure}[H]
    \includesvg[width=380pt, keepaspectratio]{Plot_SUS_Percentile}
    \caption{\gls{sus} score percentile.}
    \label{fig:plot-sus-percentile}
\end{figure}

\begin{figure}[H]
    \includesvg[width=380pt, keepaspectratio]{Plot_SUS_Time}
    \caption{Relation between total time and \gls{sus} score.}
    \label{fig:plot-sus-time}
\end{figure}

\subsection{Assignment Bias}

Considering the large differences in some of the profile variables between the groups, this subsection shows correlations between the controlled variables with greatest variance and dependent variables.
Firstly, Figure~\ref{fig:plot-correlation-education} shows a moderate to strong correlation between the relevant education and time, $r(11) = -0.27, p = 0.39$ for Group A and $r(11) = -0.69, p = 0.01$ Group B, but no correlation with the \gls{sus} score.
Secondly, Figure~\ref{fig:plot-correlation-occupation} also shows a moderate correlation between the occupation and time, $r(11) = -0.55, p = 0.06$ for Group A and $r(11) = -0.43, p = 0.17$ Group B, but there is likely no correlation with the \gls{sus} score as the data between the groups do not agree and $p > 0.1$ for both groups.
Lastly, Figure~\ref{fig:plot-correlation-age} does not show a correlation between the age group and the performance or the \gls{sus} score with $r < 0.2, p > 0.1$.

\begin{figure}[H]
    \includesvg[width=380pt, keepaspectratio]{Plot_Correlation_Education}
    \caption{Relation between between education and dependent variables.}
    \label{fig:plot-correlation-education}
\end{figure}

\begin{figure}[H]
    \includesvg[width=380pt, keepaspectratio]{Plot_Correlation_Occupation}
    \caption{Relation between between occupation and dependent variables.}
    \label{fig:plot-correlation-occupation}
\end{figure}

\begin{figure}[H]
    \includesvg[width=380pt, keepaspectratio]{Plot_Correlation_Age}
    \caption{Relation between between age and dependent variables.}
    \label{fig:plot-correlation-age}
\end{figure}
