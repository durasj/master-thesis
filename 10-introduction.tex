\chapter{Introduction}
\label{Introduction}

\todo[inline]{Please skip for now. Taken from Thesis Proposal with only minor adjustments and notes. Should be two pages that would outline content of each chapter and end with a thesis statement.}

Understanding low-level computing principles can be beneficial for a wide variety of people - from the general public interested in computers, through practising software engineers, to Computer Science students.
Rather popular material is the open-source licensed Nand2Tetris "taught at 400+ universities, high schools, and bootcamps" that explains how to build a computer from individual logic gates to high-level programming language \parencite{nand2tetrisweb}.
The material is accompanied by a desktop software that is hard to access or completely inaccessible from a new class of devices unsuited for desktop Java applications: Chromebooks that are seeing massive growth and now account for the majority of the US K12 market \parencite{Boreham_2019} \parencite{IDC_2021}, or mobile devices that account for 56\% of web traffic, up from only 6\% in 2011 \parencite{StatCounter_2021}.
The recently created web-based alternative, WepSIM, is showing promising results but takes a more complex look focusing on the CPU and instruction processing \parencite{garcia2019wepsim}.

Common problems with Massive Open Online Courses (MOOCs) leading to a high rate of dropouts include lack of time, problems adopting new systems, and bad past experience with technical problems on MOOC platforms \parencite{onah2014dropout}.
The use of MOOCs for software engineering education within higher education is argued to broaden student knowledge and is integrated by some universities in their courses and programmes \parencite{stikkolorum2014mooc}. However, MOOCs are also said to require significant time to both create and integrate \parencite{stikkolorum2014mooc}.

This thesis proposes a new web platform for learning computing principles involving logic gates via a basic individually usable tool integrated into example content.
% Can the new web platform for learning computing principles improve student task success rate compared to commonly used desktop software like that of Nand2Tetris?
