\chapter{Conclusions}
\label{Conclusions}

The work done on the thesis has shown it is feasible to reimplement the functionality of a desktop tool for learning computing principles by using web technologies while achieving efficiency, usability, and inclusivity improvements.
It was possible to do so thanks to a mature ecosystem of \gls{oss} that allowed for an efficient implementation.
The modern web \glspl{api} enabled the use with limited or even no internet connection using service workers and standalone \gls{os} integration as a \gls{pwa}.
Thanks to the client-side architecture, the resulting software does not rely on the cloud, which makes it easier to reproduce.

Participants using the proposed web implementation embedded within the learning content were able to complete the assigned tasks in $1/3$ of the time ($512 \pm190 s$ vs $1497 \pm326 s$, $\alpha=0.05$, $p<0.001$) it took participants using the desktop software.
Additionally, the mean number of times participants using the proposed software got confused was lower at $0.58 \pm 0.42$ compared to $2.83 \pm 0.85$ for the group using the existing desktop tool ($\alpha=0.05$, $p<0.001$).
This finding suggests it is possible to achieve substantial noteworthy gains by reimagining existing desktop learning tools as embedded web learning tools focused on usability.

While there was a difference of 7 points in favour of the new software (86.67 - 97th percentile vs 79.58 - 86th percentile), the result was not statistically significant ($\alpha=0.1$, $p=0.22$).
The data also suggest there is a relation between the time on task and the \gls{sus} score - at $r=-0.28$ and $r=-0.23$ for the group using the new software and the existing software, respectively.
However, the likelihood that the null hypothesis is true is high, with $p > 0.35$ for both groups.

Considering the pattern of increased reliance on devices that do not support legacy desktop applications, particularly among the 18-29 age group, K12 students using Chromebooks and tablets, and within developing countries, researchers could consider improving the inclusivity of the learning environment by designing for a wide range of devices.
Web-based learning tools, in particular, should be usable from around 98\% of devices and can be embedded within the learning content and reused in more learning materials.
The combination of inherent web advantages and a focus on usability can bring considerable efficiency improvements.
That being said, further research is needed to confirm that efficiency improvements are reproducible with different learning tools from different fields of study.
Also, more work is needed to confirm the \gls{sus} score improvements and the correlation between the performance of the participant and the \gls{sus} score.
