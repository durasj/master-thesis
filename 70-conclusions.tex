\chapter{Conclusions}
\label{Conclusions}

This thesis showed that it is feasible to reimplement the functionality of the desktop tool for learning computing principles by using web technologies while achieving device support improvements (RQ1).
It was possible to do so thanks to a mature ecosystem of \gls{oss} that allowed for an efficient implementation (RQ1).
Modern web \glspl{api}---service worker and \gls{pwa}---enabled the use with limited or even no internet connection and standalone \gls{os} integration, respectively.
Thanks to the client-side architecture, Proposed Software does not rely on the cloud, which makes it easier to reproduce.

Participants using Proposed Software embedded within the learning content could complete the assigned tasks in just one-third of the time ($512 \pm190 s$ vs $1497 \pm326 s$, $\alpha=0.05$, $p<0.001$) it took participants using Existing Software.
Additionally, the mean number of times participants using Proposed Software got confused was lower at $0.58 \pm 0.42$ compared to $2.83 \pm 0.85$ for the group using Existing Software ($\alpha=0.05$, $p<0.001$).
This finding suggests it is possible to achieve noteworthy gains by reimagining existing desktop learning tools as embedded web learning tools focused on usability (RQ2).

While there was a difference of 7 points in favour of Proposed Software (86.67 points [97th percentile] vs 79.58 points [86th percentile]), the result was not statistically significant ($\alpha=0.1$, $p=0.22$) (RQ2).
The data also suggest a relation between the time on task and the \gls{sus} score at $r=-0.28$ and $r=-0.23$ for the group using the new and existing software, respectively.
However, the likelihood that the null hypothesis is true is high, with $p > 0.35$ for both groups.

Considering the pattern of increasing reliance on devices that do not support legacy desktop applications, particularly among the 18--29 age group, K12 students using Chromebooks and tablets, and within developing countries, practitioners could consider improving the inclusiveness of the learning environment by designing for a wide range of devices.
Web-based learning tools should be usable from around 98\% of devices and can be embedded within the learning content and reused in more learning materials.
Combining inherent web advantages and a focus on usability could bring considerable efficiency improvements.
Nonetheless, further research is needed to determine the practical significance and confirm that efficiency improvements are reproducible with different learning tools from different fields of study.
Moreover, more work is needed to explore the statistically insignificant \gls{sus} score improvements as well as the correlation between the participant's performance and the \gls{sus} score.
