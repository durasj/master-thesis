\section{Learning Computing Principles}

The goal of the following section is to explore the state of remote learning and \glspl{mooc} in general concentrating on the problems or potential challenges, the most popular content available for learning low-level computing principles, and its supporting software.
Learning low-level computing principles, in this case, means introduction to topics like logic gates, chip design, computer architecture, or low-level code.
The assumed demographic is senior secondary or postsecondary students, computing professionals without a formal background in computer science, or the general public interested in computers.

\subsection{State of Remote Learning and MOOCs}
\label{sec:learning-state}

Data gathered by the \textcite{us_doe_digest_2021} show roughly 60\% of U.S. postsecondary students took at least some of their classes online in 2021.
While this number is down from the COVID-19 pandemic level of 75\% in 2020, it shows a clear upward trend compared to 36\% in 2019 and 25\% in 2012 \parencite{us_doe_digest_2021}.
More broadly, the global remote learning market was valued at about 300 to 400 billion U.S. dollars in 2022 and is expected to grow at about 10\% \gls{cagr} by some market research companies \parencites{GlobalElearning_GIA_2023}{GlobalElearning_GMI_2023}.
One of the largest \gls{mooc} providers, Coursera, reported the total number of registered users grew 61\% \gls{yoy} during the pandemic year of 2020 and 30\% the year after that \parencite{Coursera_Impact_2021}.

The effects of the accelerated widespread shift to remote learning were especially apparent during the onset of the pandemic that, at one point, \textcite{UNESCO_2022} estimates caused more than a billion children ($>70\%$) to be out of the classroom globally.
\textcite{eu_covid_learning_2023} shows disadvantaged children and children in countries with lower adoption of \gls{ict} were more likely to be negatively affected.
Taking into consideration research from \textcite{tadesse_impact_2020}, these challenges were especially evident in developing countries.
\textcite{tadesse_impact_2020} point out the lack of stable internet connection and devices suitable for remote learning as some of the challenges.
In order to support all learners, \textcite{Ali_2020} mentions, among else, that it is necessary the content is "available on a wide variety of devices and mobile friendly" and accessible despite limited bandwidth or even offline.

Although remote learning has seen a major uptick in students, \gls{mooc} dropout rates can be up to 90\% \parencite{goopio_mooc_2021}.
Common problems with \glspl{mooc} leading to a high rate of dropouts include lack of time, difficulty, and bad past experience (including technical problems) with MOOC platforms \parencite{onah2014dropout}.
\textcite{goopio_mooc_2021} mention several software-related recommendations based on their systematic review of research focused on dropouts:

\begin{itemize}
    \item Increase interactivity by providing activities like quizzes, games, or video interactions.
    \item Improve course design by using smaller chunks of content with visualisation of abstract content and real examples.
    \item Use various media formats and ensure accessibility from mobile devices and with limited internet connectivity.
\end{itemize}

From the viewpoint of teaching institutions, MOOCs are said to require significant time to both create and integrate \parencite{stikkolorum2014mooc}.
That being said, \textcite{stikkolorum2014mooc} mention the use of MOOCs for software engineering education within higher education is argued to broaden student knowledge and is successfully integrated by some universities into their courses and programmes.
Furthermore, data from the \textcite{us_doe_digest_2021} show that, at least in the U.S., online classes serve a considerably higher number of racially diverse students compared to conventional classes.

\subsection{Low-Level Computing Principles}
\label{sec:learning-principles}

One of the most popular, if not the most popular, learning materials for learning computing principles is the Nand2Tetris, "taught at 400+ universities, high schools, and bootcamps", which explains how to build a computer from individual logic gates to a working virtual computer \parencite{nand2tetrisweb}.
In the process, \textcite{nand2tetris} cover a wide variety of computing principles via practical projects:

\begin{enumerate}
    \item Construction of simple Boolean gates.
    \item Design of more advanced 16-bit chips.
    \item Incremental construction of an \gls{alu} starting with simpler chips like incrementer or half and full adder.
    \item Introduction of a clock to chips and construction of registers and \gls{ram}.
    \item Programming using simplified assembly.
    \item Construction of the final computer utilizing previously built parts.
    \item Writing of an assembler that translates custom assembly language into binary runnable on the built computer.
\end{enumerate}

The learning material is offered as a book written by \textcite{nand2tetris}, as a \gls{mooc}\footnote{Available at \url{https://www.coursera.org/learn/build-a-computer}}, and in a PDF format\footnote{Available at \url{https://www.nand2tetris.org}}.
The latter is available freely under a Creative Common Attribution-NonCommercial-ShareAlike 3.0 Unported License.
However, it does not contain additional material that builds on the mentioned projects and adds more software-related topics revolving around a Jave-like Virtual Machine, a custom high-level programming language and a standard library for a custom \gls{os}.
Regardless of which form of learning material learners choose, they are expected to use the accompanying desktop Java software.\footnote{Available at \url{https://www.nand2tetris.org/software}}

Edybrun (2021) argues the industry now focuses on web-based curricula as the web already has essential accessibility tools, and frameworks like Depth of Knowledge (DOK) lend themselves better to the web environment.
More specifically, Edybrun (2021) takes note of the development of more interactive experiences than just "text on a screen" adapted from textbooks.
For example, web-based curricula can contain interactive "embedded supports", and Edybrun (2021) mentions these should be context-sensitive.
However, the only kind of examples Edybrun (2021) provides are a tool that provides a breakdown and planning of subtasks and "virtual pedagogical agents" similar to virtual assistants like Alexa or Siri but specific to the pedagogical environment.

The recently created web-based WepSIM \parencite{garcia2019wepsim} provides microdesign, microprogramming and assembly language educational simulator and was implemented by authors as an interactive way to practice learned concepts.
Compared to Nand2Tetris, which takes a more broad approach and goes through many concepts throughout several layers of abstraction, WepSIM is focused only on a thorough simulation of the CPU and instruction processing.
\textcite{garcia2019wepsim} argue the introduction of WepSIM increased the percentage of students taking the final exam and improved grades in assembly and microprogramming exercises.
\textcite{garcia2019wepsim} also add that WepSIM is usable not only as a supplementary tool but also as a standalone learning tool thanks to the bundled examples and help material.
Notably, \textcite{garcia2019wepsim} mention students were able to use WepSim on a wide variety of devices and operating systems, from Windows-NT, Linux, and macOS to Android, iOS, and Windows-Phone.
