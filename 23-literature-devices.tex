\section{Device Type Usage}
\label{sec:device-types}

The following section focuses on the current device type usage and possible trends both in developed and developing countries.
The aim is to look at statistics both among the general population and in the education sector.
Web clients are of particular interest due to the focus of this thesis.
This section is broken down into two sub-sections for developed and developing markets, as the gathered data are considerably different.

Globally, mobile devices account for around 56\% of browser market share, up from only 11\% in 2012 \parencite{StatCounter_2023}.
The shift to mobile devices slowed significantly around the year 2017, when they reached a market share of 52\%, and there has been no apparent significant change in data since then \parencite{StatCounter_2023}.
Tablet devices, on the other hand, seemed to have reached a peak of almost 7\% in 2014, and their market share has been steadily declining since \parencite{StatCounter_2023}.

In terms of \gls{os} and screen resolution, the market is now also very fragmented \parencites{StatCounter_OS_2023}{StatCounter_Resolution_2023}.
While in the past, just 10 years ago, Windows had a commanding position, now no \gls{os} has majority of the market share and the top four makes up the same market share as Windows alone did in 2009 \parencite{StatCounter_OS_2023}.
\textcite{StatCounter_Resolution_2023} also shows how there is no major screen resolution with a market share larger than 10\% the way it used to be and how the variety of screen resolutions has been growing over time.

\subsection{Developed Countries}

Contrary to the global web traffic data, data for Europe and North America still show a gradual increase in the mobile device market share with parity reached just within the last few years \parencites{StatCounter_Europe_2023}{StatCounter_NorthAmerica_2023}.
Notably, Chromebooks haven seen a considerable growth in North America and Western Europe and accounted for the majority of the upcoming learners from US K12 market in years 2020 and 2021 \parencite{Boreham_2019, IDC_2021}.
However, tablet and Chromebook shipments have since slowed down noticeably \parencite{IDC_2022}.

In terms of internet smartphone dependency, about 15\% of U.S. adults - up from 8\% in 2013 - report a mobile device is the only way they access the internet \parencite{Pew_Research_2021}.
Importantly, this group is made up mostly of the 18-29 age group and that is the only age group where we can see a clear trend of increasing smartphone dependency \parencite{Pew_Research_2021}.
That being said, looking for example at data from \textcite{Educause_2022} collected at University of Central Florida show smartphone use for learning has not increased during the years 2016-2021 with laptops still dominating.

\subsection{Developing Countries}

In sharp contrast to Europe and North America discussed before, developing markets like Africa and Asia have a much higher market share of mobile devices at about 70\% \parencites{StatCounter_Africa_2023}{StatCounter_Asia_2023}.
Additionally, mobile devices overtook desktop devices much sooner, around the year 2015 \parencites{StatCounter_Africa_2023}{StatCounter_Asia_2023}.
Apart from that, as far as tablet devices go, there is no considerable difference we can observe with developing markets \parencites{StatCounter_Africa_2023}{StatCounter_Asia_2023}.

Similarly, while developed countries were seeing noticeable growth in Chromebook and tablet shipments, developing countries did not \parencite{Boreham_2019}.
Internet smartphone dependency is very high in emerging economies, with only about 34\% of households having access to a desktop, laptop, or tablet device \parencite{Pew_Research_2019}.
Importantly, this number varies greatly between individual developing countries, with India being at the lower end with just 11\% and Lebanon being at the high end with 57\% \parencite{Pew_Research_2019}.
That being said, smartphone ownership is prevalent mainly in the age group of 18-29, with typically less than 40\% of people aged 50+ owning a smartphone in developing countries \parencite{Pew_Research_2019}.
As far as use in education goes, data seem harder to find, but surveys say the majority of people think smartphones have a positive impact on education and educated people are more likely to own smartphones \parencite{Pew_Research_2019}.
