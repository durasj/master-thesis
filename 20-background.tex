\chapter{Background}

\todo[inline]{Outline of the covered topics should be added here.}

\section{Human-System Interaction Ergonomics}

The following section concentrates on the selected topics of Human-System Interaction Ergonomics: Accessibility, Usability, and \gls{ux}.
The selected topics are explored in relation to computer software and, where possible, web services specifically.
After introducing the concepts, this section outlines non-obvious benefits and delves into the implementation and evaluation strategies.

\subsection{Definition and Distinction}

Keeping in mind there are incompatible definitions for Accessibility provided by \gls{iso} \parencite{Wegge_Zimmermann_2007}, the specific definition chosen for this thesis describes Accessibility as the "extent to which [a service] can be used by people from a population with the widest range of user needs, characteristics and capabilities to achieve identified goals [...]" \parencite{ISO_9241-11:2018}\pdfcomment{There was no notion of pages in web versions of standards I was able to access, which is why I could not, unfortunately, reference exact pages here.}.
\textcite{ISO_10779:2020} adds that Accessibility includes but does not apply exclusively to formally disabled people.
That means Accessibility is concerned with the basic ability to utilise the software by the widest possible range of users, including disabled users \parencite{Wegge_Zimmermann_2007}.
\textcite{Wegge_Zimmermann_2007} point to existing confusion between the terms Accessibility and Usability.
Although \textcite{Wegge_Zimmermann_2007} admit these terms are related and have potential overlap, \textcite{Wegge_Zimmermann_2007} stress the importance of their distinction.
Importantly, \gls{ict} is one of the fields where Accessibility, depending on the country and sector, is mandated by the law \parencite{Wegge_Zimmermann_2007, Juergen_et_all_2020}.
For example, public sector services within \gls{eu} have to meet accessibility standards outlined in the Web Accessibility Directive, and there is an ongoing effort to extend this to the private sector \parencite{EU_Web_Accessibility}.

Usability, on the other hand, is defined by \textcite{ISO_9241-11:2018} as the "extent to which [a service] can be used by specified users to achieve specified goals with effectiveness, efficiency and satisfaction [...]".
As \textcite{Wegge_Zimmermann_2007} hint, compared to Accessibility, Usability deals with the success - "effectiveness, efficiency and satisfaction" \parencite{ISO_9241-11:2018} - of software interactions which is hard to mandate the way Accessibility is.
Therefore, there are few relevant laws and Usability is usually considered more as a competitive advantage that authors are trying to capitalise on \parencite{Wegge_Zimmermann_2007}.

Similarly to confusion between Accessibility and Usability, researchers point to issues with the distinction between Usability and User Experience \parencite{Darin_et_all_2019, Juergen_et_all_2020}.
There are multiple views on the definition of \gls{ux} and a great amount of disagreement on the topic \parencite{Juergen_et_all_2020}.
\textcite{ISO_9241-11:2018} defines \gls{ux} as "user’s perceptions and responses that result from the use and/or anticipated use of [a service]".
Compared to the sole satisfaction that was mentioned to be a part of Usability and \textcite{Darin_et_all_2019} argue is incorrectly used as, and often only, \gls{ux} measurement, \gls{ux} deals with "users’ emotions, beliefs, preferences, perceptions, comfort, behaviours, and accomplishments" \parencite{ISO_9241-11:2018}.
Additionally, \gls{ux} is concerned with a broader timeline - the time spent performing the task \emph{and} the time before and after that \parencite{Juergen_et_all_2020,ISO_9241-11:2018}.
Considering the definition by \textcite{ISO_9241-11:2018}, we can look at \gls{ux} as a more complex superset of Usability \parencite{Juergen_et_all_2020}.

In short, for the purposes of this thesis, Accessibility is the basic ability to use the software regardless of the user's various limitations, Usability is the extent to which it can be used to achieve delineated goals successfully, and \gls{ux} is a more complex extension to Usability also concerned with impressions both during and outside of the use of software.
The following subsections always refer to these ideas as Accessibility, Usability, and \gls{ux}, even if cited literature refers to them differently.
\pdfcomment{Is it okay to include this paragraph? Should the used simplified definitions be rather moved to a glossary? And part of it moved to the section introduction above?}

\subsection{Benefits}

\textcite{Juergen_et_all_2020} point to their earlier research results\pdfcomment{Should I cite such research directly if I did not read it directly, and I think it would derange the reader? We are talking about 4 earlier papers on the topic.} that indicate implementing Accessibility on the web could provide benefits to users other than the usual primary target of Accessibility enhancements - disabled users.
This notion is seconded by \textcite{Vanderheiden_2000}, who mentions that others in similar challenging situations can benefit from Accessibility as well.
\textcite{Edyburn_2010} provides terminology identifying these two groups of users: primary and secondary beneficiaries. \textcite{Edyburn_2021} mentions that Accessibility can improve the ability to use the software for both primary beneficiaries - disabled students - and secondary beneficiaries - all other students - in the educational setting.

Examples of secondary beneficiaries in relation to the web are mentioned by \textcite{WAI_Intro}:

\begin{itemize}
    \item people using different devices with smaller screens or "different input modes" \parencite{WAI_Intro},
    \item people challenged by limitations introduced by ageing,
    \item temporarily limited people - e.g. by a "broken arm or lost glasses [...or a] bright sunlight" \parencite{WAI_Intro}, or conditions that do not allow audio playback,
    \item and people limited by their internet connection - i.e. latency or bandwidth.
\end{itemize}

However, \textcite{Juergen_et_all_2020} also mention that benefits from concepts like "using easy language" \parencite[p. 1210]{Juergen_et_all_2020} may not be significant enough.
Additionally, as far as \gls{wai} is concerned, we need to point out the inherent bias.\pdfcomment{Is this something I am allowed to mention here?}

\subsection{Implementation}

According to \textcite[p. 296]{Wegge_Zimmermann_2007}, there are three main approaches to implement the Accessibility requirements:

\begin{itemize}
    \item Universal Design - designing software to be usable without any modifications by the widest range of users.
    \item Adaptive Design - designing software to be adaptable to different types of users.
    \item Interoperability with Assistive Technology - designing software to work with existing assistive software.
\end{itemize}

\textcite{Wegge_Zimmermann_2007} warn about the downsides of Universal Design: the ability to hinder the experience of the majority, stigmatisation, and implementation difficulties due to conflicting requirements.
In contrast to that, implementation of Adaptive Design allows to opt-in to alternative, independent representations without influencing other groups of users \parencite{Wegge_Zimmermann_2007}.
Similarly, assuming the software follows the relevant standard \gls{api}, Interoperability with Assistive Technology allows selected users to use their existing tools, e.g. screen readers \parencite{Wegge_Zimmermann_2007}.
\textcite{Edyburn_2021} mentions Assistive Technologies like speech-to-text and text-to-speech in relation to learning and argues they are already available on most platforms and can be targeted at both types of beneficiaries.

\textcite{Juergen_et_all_2020} mention that the web has received a significant amount of attention in regards to Accessibility thanks to its perceived importance.
\gls{wai} is recognised as the relevant source of information used to develop and verify the Accessibility of web services \parencite{WAI_Intro}.
\gls{wcag} by \gls{wai} are adopted around the world, including in the law \parencite{WAI_Policies}, an example of which is the \gls{eu}['s] Web Accessibility Directive \parencite{EU_Web_Accessibility}.

\todo[inline]{The content below should be revised and finished and I have included it only for the context.}

According to \parencite{WAI_Topics} there are two kinds of Accessibility requirements: technical, which are mostly fulfilled by properly utilising available \gls{api}[s]; and interaction/visual.

\subsection{Evaluation}

\textcite{Wegge_Zimmermann_2007} mention Usability, in contrast to Accessibility, is not typically tested by adherence to standards but by testing and analysing the impact on the end-users.
Usability testing is done by focusing on specific roles and tasks \parencite{Wegge_Zimmermann_2007}.
\textcite{Edyburn_2021} adds we can monitor subjective measures like decreased frustration or objective measures like increased productivity during the Ergonomics research and practice for the educational setting.

\todo[inline]{Summarise tools and methods mentioned by \parencite{Darin_et_all_2019}, strategies by \parencite{Wegge_Zimmermann_2007}, and approaches by \parencite{Juergen_et_all_2020}. Mention comparative UX studies explicitly and guidelines on the size of the group.}

\todo[inline]{Specifically for web - mention methods outlined by \parencite{WAI_Evaluate}.}


\section{Learning Low-Level Computing Principles}

Edybrun (2021) argues the industry now focuses on web-based curricula as the web already has essential accessibility tools, and frameworks like Depth of Knowledge (DOK) lend themselves better to the web environment. More specifically, Edybrun (2021) mentions developing more interactive experiences than just "text on a screen" adapted from textbooks. Part of the web-based curricula can be interactive "embedded supports". Edybrun (2021) mentions these should be context-sensitive. However, the only kind of examples Edybrun (2021) provides are a tool that provides a breakdown and planning of subtasks and "virtual pedagogical agents" similar to virtual assistants like Alexa or Siri, but specific to the pedagogical environment.

\section{Device Usage Trends}

\todo[inline]{Work in progress.}

Chromebooks are seeing massive growth and now account for the majority of the US K12 market \parencite{Boreham_2019, IDC_2021}
Mobile devices account for 56\% of web traffic, up from only 6\% in 2011 \parencite{StatCounter_2021}.

\todo[inline]{Overlap between Web Accessibility and Mobile Web.}

\section{Creating Open-Source Web Application}


