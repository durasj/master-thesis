\chapter{Background}

This thesis will explore material for learning computing principles and accompanied software, the state of the MOOCs and their shortcomings, and current best practices on creating and maintaining open-source software.

The mentioned exploration will feed into the design, development, and evaluation of a new web application.
There are several deliverables expected at the end of the research:

\begin{enumerate}
    \item[A] Simple web platform accessible from all major devices without any setup, including Chromebooks, tablets, and smartphones. It will feature the ability to sign in, retain the progress, and share the state of individual tools.
    \item[B] Reimagined web version of Hardware Simulator, Hack Computer, and Hack Assembler from Nand2Tetris running solely on the client-side web. These will be usable standalone and embedded.
    \item[C] Content management for the web platform via Markdown files hosted at Github with the first half of the CC-licensed Nand2Tetris content as a starting point.
    \item[D] Primary data on selected objective User Experience (UX) metrics useful for judging usability in and out of classrooms - completion rate, time on task, and the number of problems/frustrations. 
\end{enumerate}

These deliverables should be useful to all three groups of people mentioned in the \nameref{Introduction}. Specifically, these deliverables will be usable as:

\begin{itemize}
    \item Easily accessible alternative software for the Nand2Tetris course available at Coursera \footnote{Build a Modern Computer from First Principles: From Nand to Tetris (Project-Centered Course), URL: \url{https://www.coursera.org/learn/build-a-computer}}. The software will be shared with the community and proposed as an alternative to users struggling to run the original software, e.g., Chromebook, tablet, smartphone, Windows S users, and users lacking permissions or experience to install Java.
    \item Individual tools that can be integrated into traditional education. Since the tools will be kept as simple as possible, they should be usable with relatively low time investment.
    \item A standalone platform for self-education of computing principles with the starting content and ability to extend it as people see fit, thanks to its open nature.
    \item An example showing attainable accessibility and efficiency improvements from moving desktop education tools to the web.
\end{itemize}
