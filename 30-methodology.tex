\chapter{Methodology}

\todo[inline]{Taken from Thesis Proposal with only minor adjustments and notes.}

The main paradigm used for this thesis is Design Science Research (DSR), as currently outlined by \textcite{brocke2020designscience}.
It uses existing theories, frameworks, and models to produce and evaluate a new artefact - instantiation, as categorised by \textcite{hevner2004designscience}.

The evaluation of the artifact consists of two parts:

\begin{itemize}
    \item \textbf{Automated Tests}
    It is assumed it is possible to achieve feature parity with the selected three tools.
    To evaluate this and confirm the feasibility of porting to the client-side web, the author will prepare two classes of automated tests: end-to-end and unit tests.
    End-to-end tests will focus on the ability to perform equivalent actions (feature-parity) and unit tests, among else, to confirm various inputs, including the Nand2Tetris assignments from all six weeks, can be used and produce correct output (correctness).
    \item \textbf{Primary Quantitative Research}
    The central hypothesis is that the increased efficiency from using the web instead of the desktop will increase the student task success rate.
    This will be evaluated by conducting a comparative study.
    The participants will be tasked to perform a simple exercise of implementing chips like AND and NAND or XOR and XNOR in Nand2Tetris and the newly developed platform.
    The order of implemented gates and used programs will be randomised, and participants will be from different backgrounds corresponding to three target demographics mentioned in the \nameref{Introduction} - Computer Science students, practitioners with/without formal education, and people interested, but not directly involved, in Computer Science.
    The tasks will be performed in their home environment corresponding to currently widespread distance learning or self-learning and will include the setup of the software.
    Collected data will include objectively measurable UX task metrics: completion rate, time on task, and the number of problems/frustrations.
\end{itemize}

Collected knowledge will be interpreted and described as part of the discussion.
It is expected unstructured qualitative data could be generated in the process, e.g., the experience of participants using and integrating the system and their feedback.
If that happens to be the case and the data will be significant enough, they will be documented as part of the discussion. However, this thesis will not delve into qualitative data.
